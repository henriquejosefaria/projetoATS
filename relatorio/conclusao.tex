\vspace{1cm}
\section{Conclusão}
\hfill\newline
\par Neste trabalho começamos por examinar os problemas do código que nos foi entregue, isto foi feito com recurso ao SonarQube. Após este procedimento passamos a fazer o refactoring da aplicação fazendo uso de ferramentas como o auto-refactor do inteliJ e utilizando as dicas oferecidas pelas regras do SonarQube e o IDE do Eclipse. O refactoring começou por ser realizado no bugs, seguindo-se as vulnerabilidades e os code smells. No fim das correções ficamos com apenas 1 code smell. \newline
\par Posteriormente passamos aos testes da aplicação refactorizada. Primeiro executamos um teste unitário sobre a classe \textit{Cliente.java}, este teste serviu para demonstrar os nossos conhecimentos sobre a aplicação de testes a um código. De seguida usamos o \textit{EvoSuite} para gerar automaticamente testes sobre todas as classes do nosso projeto. E para correr os testes verificando a cobertura destes face ao código escolhemos a opção correr testes com cobertura usando JaCoCo como o Coverage runnner.\newline
\par Por fim criamos um ficheiro chamado quick.hs para gerar o novo ficheiro de input do programa e posteriormente fizemos a análise de desempenho da aplicação, antes e depois do refactoring.\newline
\par Nas imagens finais podemos ver que da aplicação original para a aplicação refatorizada houve uma melhoria em termos de energia consumida, especialmente quanto maiores os ficheiros, nos quais se obtem menos de metade do consumo face a aplicação original.\newline
Em suma, existem vantagens em fazer o refactoring de um código, pois este pode permitir poupar energia e ser mais rápido e eficiente que o original. Para além disso, este possibilita uma mais facil manutenção de código futuro e correção de partes de código que podem levar a um mau desempenho ou a uma falha da aplicação no futuro.

