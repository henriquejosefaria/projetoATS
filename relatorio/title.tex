

\title{UmCarroJá: Análise e Teste de Software}

\author{Henrique Faria A82200 \and Sandra Baptista PG35390}


\institute{Departamento de Informática, Universidade do Minho}


\maketitle

% Abstract
\begin{abstract}
Neste trabalho propusemo-nos a analizar e testar o software feito no ambito da disciplina de \textit{Programação Orientada a Objetos}.\newline
Este relatório encontra-se estruturado em 4 secções: \textit{Qualidade do Código Fonte, Refactoring da Aplicação, Teste da Aplicação e Análise de Desempenho da Aplicação}.\newline
Foram também utilizadas as seguintes ferramentas para realizar o trabalho:\newline
Eclipse, SonarQube, JStanley, IntelliJDEA, RAPL, EvoSuite.
\keywords{Code Smells \and Technical Debt \and Eclipse \and SonarQube \and JStanley \and IntelliJDEA \and RAPL \and EvoSuite}
\end{abstract}

%Glossário

%Code smells não são bugs e também não estão tecnicamente incorretos. No entanto estes indicam fraquezas no design de uma aplicação que podem comprometer-la quer diminuindo o progresso do desenvolvimento da mesma quer provocando bugs ou falhas no futuro. Maus code smells podem provocar resultados adversos aos que se pretendem na aplicação conhecidos como technical debt.\newline
%Technical debt é um conceito de software que reflete o custo adicional implicito de modificação de código futuro como consequência da utilização de uma solução limitada mas facil de implementar em vez de implementar uma um pouco mais trabalhosa mas que não demande refazer ou reimplementar código no futuro.


